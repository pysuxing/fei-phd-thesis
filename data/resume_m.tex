\begin{resume}

  \section*{发表的学术论文} % 发表的和录用的合在一起

  \begin{enumerate}[{[}1{]}]
  \addtolength{\itemsep}{-.36\baselineskip}%缩小条目之间的间距,下面类似
  \item 第一作者. Galaxyfly: A Novel Family of Flexible-Radix Low-Diameter Topologies for Large-Scales Interconnection Networks. in Proceedings of the 2016 International Conference on Supercomputing(ICS), 2016, pp. 24:1–24:12.(EI检索号:20163002626759,CCF B 类会议)
  \item 第一作者. An Efficient Label Routing on High-Radix Interconnection Networks. in Proceedings of the 2017 International Conterence on Parallel and Distributed Systems (ICPADS), 2017, pp.596-603.(EI检索号:20182405316608,CCF C类会议)
   \item 第一作者. Exploring the Galaxyfly Family to Build Flexible-Scale Interconnection Networks. (已投稿至 Transactions on Computers (TC),CCF A类期刊).
   \item 第一作者. Exploring the Galaxyfly Family to Build Flexible-Scale Interconnection Networks. (已投稿至 ACM Transactions on Architecture and Code Optimization(TACO),CCF B类期刊).
  \item 第一作者. Paleyfly:一种可扩展的高速互连网络拓扑结构[J].计算机研究与发展, 2015. 52(6):1329-1340.(EI检索号:20152901043559)
  \item 第一作者. SuperStar:一种可扩展高阶互连拓扑结构[J]. 计算机工程与科学, 2014. 36(6):39-46 .
  \item 第一作者. 高阶互连网络拓扑结构性能分析与研究[J]. 计算机工程与科学, 2013. 35(11):111-118 .
  \item 第二作者.一种新型混合互连网络拓扑结构的分析与优化[J].计算机工程与科学, 2014. 36(12):2400-2409.
  \item 第五作者. RoB-Router: Low Latency Network-on-Chip Router Microarchitecture Using Reorder Buffer. in Proceedings of the 2016 IEEE Hot Interconnects (HOTI), 2016, pp.68-75. (EI检索,检索号:20170503305832)
  \item	第四作者. CCAS: Contention and congestion aware switch allocation for network-on-chips. in Proceedings of the 34th IEEE International Conference on Computer Design (ICCD), 2016, pp.444-447. (EI检索,检索号:20165203166084)
  \item 第四作者. 低延迟路由器中高效开关分配机制的实现与评测. [J]. 湖南大学学报(自然版), 2015. 42(4): 78-84. (EI检索,检索号:20151900831996)
  \item 第四作者. 一种多级无缓存高阶路由器的设计与实现. [J]. 计算机工程与科学, 2017, 39(2):245-251.
  \item 第四作者. 高阶路由器结构研究综述. [J]. 计算机工程与科学, 2016, 38(8):1517-1523.
  \end{enumerate}

  \section*{研究成果} % 有就写,没有就删除
  \begin{enumerate}[{[}1{]}]
  \addtolength{\itemsep}{-.36\baselineskip}%
  \item  一种支持大规模、全光互连的光交换机: 中国,
     CN105681932A (中国专利公开号).
  \end{enumerate}


\section*{参与的科研工作}

  \begin{enumerate}[{[}1{]}]
  \addtolength{\itemsep}{-.36\baselineskip}%
  \item 国防预研项目:大规模互连网络模拟仿真技术
  \item 国家重点研发计划项目:基于数据流的大数据分析系统
  \item 国家“863”计划重大项目:天河新一代高性能计算机系统研制
  \item 国家重点研发计划项目:百亿亿次E级超算原型系统研制
  \end{enumerate}
\end{resume}
