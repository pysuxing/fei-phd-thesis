
%%% Local Variables:
%%% mode: latex
%%% TeX-master: "../main"
%%% End:

\begin{ack}

四年半的博士时光匆匆而过,在论文即将完成之际,回首四年半来的学习、工作和生活,中间有过迷茫、有过痛苦、有过喜悦,但此刻充满我心中的是深深的感激之情,感谢那些给予
我关心和帮助的老师、同学和朋友,你们的关心与帮助不仅使我能够顺利完成博士学业,而且是我今后一生受用不尽的财富。言语已经不能完全表达我的感激之情,我需要在今后工作、生活中用行动来报答。但是我还是想在这里写下一些感激的文字,以表达我衷心的感恩之情!

感谢我的导师廖湘科研究员! 无论是学习、科研工作,还是生活中的大事小情,亦或做人处世,您都给了我无微不至的关怀和帮助。您勤奋严谨的治学态度、对科学问题的深邃洞察力、对学术方向的高瞻远瞩以及对新领域积极探索的精神,都是我不断成长的力量源泉,是我终生学习的榜样。您指导了我的人生之路,事业之路,是我用今后一生都报答不尽的。特别感谢我们的谢师母!谢师母是慈母,也是良师,是益友!您温暖的笑容,让我们感受到了慈母般的关怀!

感谢董德尊研究员!您渊博的学识、敏锐的洞察力、勤奋进取的科研精神、严谨的治学态度以及平易近人、待人友善的生活作风,无时无刻不在熏陶和感染着我,并将让我终身受益。每当碰到困难与问题向您求教时,总能获得您宝贵的建议和解决问题的思路,让我受益匪浅。您带我走进了高性能互连网络这一研究领域,认识到这一领域在国际上的发展趋势。在您的指导下,完成了多篇学术论文。这些论文的选题、研究以及最后的撰写,都凝结了您的心血。您还让我接触到了多位国际知名学者,开阔了我的视野。感谢嫂子姬少丽!嫂子对董师兄工作的支持和理解,让我们充满敬意!

感谢Jose Duato教授!在我访学西班牙瓦伦西亚理工大学期间以及通过邮件讨论给我在课题上提供了很大的帮助和宝贵的意见,并在悉心指导下,完成了相关问题学术论文的撰写。

感谢我所在的教研室的领导和老师们!你们为我提供了良好的学习科研环境,并给予我大量具有创造性和启发性的指导!

感谢杨沙洲师兄和李姗姗师姐!硕士入学,是他们开启了我的学术科研之旅,是他们治学严谨的科研精神感染了我!

感谢互连研究团队的成员!吴际、保金桢、李存禄、柴燕涛、杨明英、方磊、李彬、娄辉、朱成阳、杨文祥、徐叶茂、师凯、杜记伟、张庆安、张敏、赵静月、邹祥喜、祝雅正、张凡、吴君楠、孙凯旋、黄山、朱佳平、杨天野、魏子昊、吴克、金康、周择嘉、张鑫、白洋,和你们一起学习、奋斗,营造了和谐、团结、奋进的实验室氛围,和大家一起的奋斗的时光将成为美好的回忆!也要谢谢从事其它研究工作的师兄弟们!他们是鲁晓佩、刘晓东、郑思、林彬、朱浩、张菁、郭勇、黄訸、任静、范小康、申彤、张峰、徐尔茨、崔英博、贾周阳、周书林、顾祥、徐向阳、郦旺、张晓雨、李云峰、刘晋宇、池书琪、王腾、张元良、何浩辰。

感谢多年来一直支持和陪伴的好友和同学!感谢206大家庭王倩、丰瑶、杨静、史乔给予我的温暖!感谢郭辉、陈辰、陆华俊一起游历祖国大江南北!感谢宋省身、陈洪义一起探索美食运动和时事!感谢曾皓、陈呈、高涛一起探讨人生学术和理想!感谢邻居李豪和岑阳夫妇!感谢14级2班的全体成员!感谢组委会的兄弟姐妹们!

感谢博士生队的队领导们对我的关怀和帮助,他们默默践行着为学员创造良好环境的承诺,为我们的科研和生活提供了稳固的保障。
感谢本文及相关小论文的所有评审老师!你们的评审意见为论文的改进与完善提供了很大的帮助!

感谢我的父母其他亲人! 没有你们的支持、理解和鼓励就没有我的一切。

感谢我的爱人苏醒,相知相遇一路走来,非常幸运一起携手经历博士生涯。愿我们
共同努力奋斗,一起迎创新的未来!

最后,再次诚挚地感谢所有给予我关心与帮助的以及未能列出名字的领导、老师、亲人和朋友们,感谢你们!



\end{ack}
