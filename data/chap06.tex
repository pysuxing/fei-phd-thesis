\chapter{结论与展望}

\section{工作总结}

 高性能互连网络是高性能计算系统的重要组成部分。随着应用程序对计算能力需求的不断增长,
 高性能计算机系统规模不断增大,高性能互连网络的性能越来越成为制约系统整体性能的重要因素。
 因此,设计面向未来更大规模计算系统的高性能互连网络是很有必要的。
 大规模低直径网络的是当前与未来高性能互连网络的主流发展方向,
 在大规模低直径互连网络设计中,需要考虑很多限制因素,如性能需求、物理封装、
 器件约束、路由算法、能耗开销等问题。因此,开展大规模低直径高性能互连网络
 的相关研究工作对高性能计算系统发展有着重要的意义。

 本文针对网络拓扑结构如何在当前物理器件约束下应对E级计算系统网络规模需求
 与物理布局布线的复杂度,
 路由算法如何解决缓存资源利用率低等问题,
 分别对规模灵活、易于布线的网络拓扑结构设计以及
 可高效利用缓存资源的路由算法设计等挑战性问题进行深入研究。本文的工作主要包括:

  (1)针对在E级计算的挑战下,
  因路由器端口数约束造成当前网络拓扑结构规模受限、灵活性低、网络性能不足等问题,
  提出一种高性能互连网络新型拓扑结构Galaxyfly。
  Galaxyfly基于代数图论有限域的方法构建,
  其降低了对高阶路由器端口数的要求,
  可以使用较少的端口数满足E级计算系统的规模需求,
  在保持低直径的情况下,Galaxyfly可以达到网络规模与二分带宽的灵活折中。
  %针对Galaxyfly结构,分析了可构造的配置并且评估了最短路径数量。
  利用其代数图论的性质,设计了用于Galaxyfly的拥塞敏感的路由算法。
  分别从性能、成本和能耗三方面与其他新型高性能互连网络拓扑结构
  进行了实际物理布局的模拟和分析比较,
  结果表明,相比其他拓扑结构,Galaxyfly在不同的路由算法以及典型的通信模式下,
  能够展现更优的性能,是一个适合构建E级计算系统的新型高性能互连网络拓扑结构。

  (2)针对在E级计算的挑战下,
  当前高性能互连网络的网络性能、可维护性以及物理封装等方面的不足,
  提出一种适合使用多芯光纤的高性能互连网络新型拓扑结构Bundlefly。
  Bundlefly是一个低直径、可灵活扩展并且适合采用多芯光纤作为机柜间连线的拓扑结构。
  随着集成光模块板的发展,一根多芯光纤可以替代一捆传统的单芯光纤,
  不仅可以降低光纤的使用成本还可以提高光纤的可维护性。
  在保证网络直径只有3的前提下,
  Bundlefly不仅能够充分利用多芯光纤来提高机柜间的通信带宽,
  还能降低对高阶路由器的端口数要求以支持E级系统的规模。
  分析和模拟的结果表明,与其他新型高性能互连网络拓扑结构相比,
  Bundlefly表现出更优的性能。

  (3)针对目前高性能互连网络自适应路由算法对虚拟通道数量要求过高以及
  缓存资源利用不均衡的问题,
  提出了一种标签路由算法Label-based Routing(LBR)。
  LBR通过协同设计路由器微体系结构中的输入缓冲区模块和路由计算模块,
  将路由计算引入缓冲区模块,根据网络状态对路由报文做标记。
  有效实现完全自适应路由的前提下,LBR不仅降低了死锁避免对虚拟通道数的需求,
  还使缓存资源利用更加均衡。
  通过在Dragonfly上模拟不同的路由算法,在大部分通信模式下,
  LBR的性能超出其他路由算法约10\%-35\%。
%% (1)针对在E级计算的挑战下,因路由器端口数约束使当前高性能互连网络的灵活性及网络性能方面的不足,提出一种灵活端口数的
%% 高性能互连网络新型拓扑结构Galaxyfly。Galaxyfly利用代数图论有限域的方法构造而成。在保持低直径的情况下,
%% Galaxyfly可以达到网络规模和二分带宽的灵活权衡。其降低了对高阶路由器端口数的要求,
%% 可以使用较少的端口数去构建E级计算系统的网络规模。针对Galaxyfly结构,不仅分析了可构造的配置并且评估了最短路径数量。
%% 利用其代数图论的性质,设计了拥塞敏感的路由算法。与其他新型高性能互连网络拓扑结构分别从性能、
%% 成本和能耗三方面进行了实际物理布局的模拟和分析比较。结果表明Galaxyfly相比其他结构,在不同的路由算法以及典型的通信模式下,
%% 能够展现更优的性能,是一个适合构建E级计算系统的新型高性能互连网络拓扑结构。

%% (2)针对在E级计算的挑战下,目前高性能互连网络的网络性能、可维护性以及物理封装方面的不足,
%% 提出一种适合使用多芯光纤的高性能互连网络新型拓扑结构Bundlefly。Bundlefly是一个低直径、
%% 可灵活扩展并且适合采用多芯光纤连接机柜之间的拓扑结构。随着集成光模块板的发展,
%% 一根多芯光纤可以替代一捆传统的单芯光纤,不仅可以降低光纤的使用成本还可以提高光纤的可维护性。
%% 虽然Bundlefly的网络直径只有3,但是其不仅能够充分利用多芯光纤来提高机柜间的通信带宽
%% 还能降低高阶路由器的端口数的要求来支持E级系统的可扩展性。通过分析和模拟,
%% 比较了Bundlefly和其他新型高性能互连网络拓扑结构,
%% Bundlefly能够完成更好的性能。

%% (3)针对目前高性能互连网络自适应路由算法对虚拟通道数量要求高以及缓存资源利用均衡的不足,
%% 提出了一种标签路由算法Label-based Routing(LBR)。
%% LBR通过协同设计路由器微体系结构里的输入缓冲区模块和路由计算模块,将路由计算引入缓冲区模块,
%% 根据网络状态对路由报文做标记。LBR不仅降低了死锁避免政策对虚拟通道的需求,
%% 还均衡使用缓存资源并有效实现完全自适应路由。
%% 通过模拟在Dragonfly结构上评估了LBR的性能并与其他拓扑结构进行了对比。
%% 实验表明,在大部分通信模式下,LBR优于别的路由算法近10\%-35\%。

\section{研究展望}

基于本论文已经取得的成果,在后续的工作中,
我们将进一步在以下几个方面进行完善并开展研究:

(1)高性能互连网络新型拓扑结构的高效路由算法。
在新型高性能互连网络中,由于最短路径数量有限,常使用非最短路径作为自适应路由的选择。
因此,不同的通信模式对网络性能影响很大,
尤其对一些密集的通信模式。
随着系统规模的增大,应用负载的多样化,传统的自适应路由算法也不能很好的应对。
在进一步工作中,我们将通过分析不同的应用负载,提取通信模型,
有针对性的设计基于通信模型的高效路由算法以减少网络拥塞,做到负载均衡。

(2)以减少头阻塞为优化目标的缓冲区分配策略。
随着系统规模的增加,大量不同目的地的报文会存放在同一个缓冲区队列中,这样就
容易造成头阻塞现象,从而影响网络性能。在进行缓冲区资源分配的时候要综合考虑
不同目的地的报文如何存放以便减少头阻塞的发生。目前,已有相关研究初步探讨大规模
低直径的典型拓扑结构的头阻塞问题,如Y$\'{e}$benes等人探讨Slim Fly结构上减少
头阻塞的优化问题。相关问题还存在很大的研究空间值得深入研究。


